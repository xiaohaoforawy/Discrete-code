\documentclass[12pt,a4paper]{article}
%\usepackage{ctex}
\usepackage{graphicx}
\usepackage{listings}
\usepackage{abstract}
\usepackage[colorlinks,linkcolor=blue]{hyperref}


\begin{document}


%\tableofcontents
\title{Analysis of Numpy Computing Efficiency}

\author{\href{http://awy.ink/contact}{Wei Xiao}\thanks{I am grateful to the PLPP teacher for supervise me during this course.}}

\date{\small \today}

\maketitle

\begin{abstract}

The NumPy system is an open source, high-performance numerical computing extension package for the Python language.
 Its performance usually affects the efficiency of the entire program. 
 We analyze the memory management rules of Numpy on a Python virtual machine, the Python virtual machine on the instruction set, the operation to find the optimal calculation efficiency.
We use the Python built-in tools such as cProfile, \%timeit, and memory\_profiler to analyze the performance of the function. We measure the different computational efficiency of different function statements.
Finally based on the above analysis gives a part of the recommendations to improve the efficiency of the implementation\footnote{This isn't the last title.With more study in high performance of Python.I may change to another title.}
\end{abstract}


\large\textbf{keywords} &nbsp;{ &nbsp;Numpy, Optimization,Efficiency}
\newpage

\section{Introduction}
The introduction section mainly describes the python language, the main problem of low efficiency, the cost of the type judgment, memory operation, etc.

\section{Numpy System}

\subsection{Environment}

\subsubsection{Python virtual machine}

Introducting the basic concepts of Python virtual machine\cite{a1}...
\subsubsection{Numpy background}
The Numpy is developed by C,Fortran,shell... 

\subsection{Memory Rule}

\subsubsection{List}
Application and release rules of memory in the Numpy list\cite{a2}
\subsubsection{tuple}
Rules for the management of elements in Numpy tuples.

\begin{figure}[h]
	\centering
	\includegraphics[width=0.7\textwidth]{a.jpg}
	\caption{The difference of range and xrange}
\end{figure}

\section{Optimization of the status}

\subsection{Computing optimization}
there are another reason of using CPU
\begin{equation}
	E=mc^2
\end{equation}
to be honest, python have many failure.in the next code\cite{a1,a2}
\begin{lstlisting}[language=python]
	class Solution:
    def findMedianSortedArrays(self, nums1, nums2):
        """
        :type nums1: List[int]
        :type nums2: List[int]
        :rtype: float
        """
        sl=len(nums1)+len(nums2)
        for i in range(sl):
            if len
            if(nums1[0]<nums2[0])
\end{lstlisting}
\subsection{Memory Optimization}

balabala...

\subsection{Algorithm Optimization}

balabala...

\section{a more useful Optimization}

\subsection{conclusion}
......

\subsection{Result analysis}
.......
\newpage
\begin{thebibliography}{3}
	\bibitem{a1} High Performance Python by Micha Gorelick and Ian Ozsvald (O’Reilly). Copyright 2014 Micha Gorelick and Ian Ozsvald, 978-1-449-36159-4. 
	\bibitem{a2} The NumPy Array: A Structure for Efficient Numerical Computation. Computing in Science and Engg. 13, 2 (March 2011), 22–30
	\bibitem{a3} Bergstra, J., Breuleux, O., Bastien, F., Lamblin, P., Pascanu, R., Desjardins, G., Turian, J., Warde-Farley, D., Bengio, Y., 2010. Theano: a CPU and GPU math expression compiler. In: Proceedings of the Python for Scientific Computing Conference (SciPy), June 30–July 3, Austin, TX.
\end{thebibliography}

\end{document}